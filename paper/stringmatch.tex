%%%%%%%%%%%%%%%%%%%%%%%%%%%%%%%%%%%%%%%%%
% Journal Article
% LaTeX Template
% Version 1.1 (25/11/12)
%
% This template has been downloaded from:
% http://www.LaTeXTemplates.com
%
% Original author:
% Frits Wenneker (http://www.howtotex.com)
%
% License:
% CC BY-NC-SA 3.0 (http://creativecommons.org/licenses/by-nc-sa/3.0/)
%
%%%%%%%%%%%%%%%%%%%%%%%%%%%%%%%%%%%%%%%%%

%----------------------------------------------------------------------------------------
%	PACKAGES AND OTHER DOCUMENT CONFIGURATIONS
%----------------------------------------------------------------------------------------

\documentclass[twoside]{article}


\usepackage{amsmath}
\usepackage{etex}
\usepackage{tikz}
\usetikzlibrary{trees}
\usepackage{syntax}
\usepackage{listings}
\usepackage{color}

\usepackage[utf8x]{inputenc}
\usepackage[ngerman]{babel}

\usepackage[sc]{mathpazo} % Use the Palatino font
\usepackage[T1]{fontenc} % Use 8-bit encoding that has 256 glyphs
\linespread{1.05} % Line spacing - Palatino needs more space between lines
\usepackage{microtype} % Slightly tweak font spacing for aesthetics

\usepackage[hmarginratio=1:1,top=32mm,columnsep=20pt]{geometry} % Document margins
\usepackage{multicol} % Used for the two-column layout of the document
\usepackage{hyperref} % For hyperlinks in the PDF

\usepackage[hang, small,labelfont=bf,up,textfont=it,up]{caption} % Custom captions under/above floats in tables or figures
\usepackage{booktabs} % Horizontal rules in tables
\usepackage{float} % Required for tables and figures in the multi-column environment - they need to be placed in specific locations with the [H] (e.g. \begin{table}[H])

\usepackage{lettrine} % The lettrine is the first enlarged letter at the beginning of the text
\usepackage{paralist} % Used for the compactitem environment which makes bullet points with less space between them

\usepackage{abstract} % Allows abstract customization
\renewcommand{\abstractnamefont}{\normalfont\bfseries} % Set the "Abstract" text to bold
\renewcommand{\abstracttextfont}{\normalfont\small\itshape} % Set the abstract itself to small italic text


\setcounter{secnumdepth}{5}
\usepackage{titlesec} % Allows customization of titles
\titleformat*{\section}{\LARGE\bfseries}
\titleformat*{\subsection}{\Large\bfseries}
\titleformat*{\paragraph}{\large\bfseries}
% Change the look of the subsection titles

\usepackage{fancyhdr} % Headers and footers
\pagestyle{fancy} % All pages have headers and footers
\fancyhead{} % Blank out the default header
\fancyfoot{} % Blank out the default footer
\fancyhead[C]{August 2015 $\bullet$ Freie Universität Berlin} % Custom header text
\fancyfoot[RO,LE]{\thepage} % Custom footer text

%----------------------------------------------------------------------------------------
%	TITLE SECTION
%----------------------------------------------------------------------------------------

\title{\vspace{-15mm}\fontsize{24pt}{10pt}\selectfont\textbf{Suffixarrays:
    Einführung und Anwendungen
}} % Article title

\author{
\large
\textsc{André Röhrig} \\ % Your name
\normalsize \href{mailto:andre.roehrig@fu-berlin.de}{andre.roehrig@fu-berlin.de} % Your email address
\vspace{-5mm}
}
\date{}


%----------------------------------------------------------------------------------------

\begin{document}

\maketitle % Insert title

\thispagestyle{fancy} % All pages have headers and footers
%----------------------------------------------------------------------------------------
%	ABSTRACT
%----------------------------------------------------------------------------------------

\begin{abstract}Das Suffixarray ist eine Struktur, welche
  die Suffixe eines Strings lexikografisch sortiert anhand
  ihrer Indizes angibt. Hier werde ich zum Zwecke der
  Einführung die grundlegenden Ideen dieser Struktur
  darstellen, den Konstruktionsalgortihmus von Manber und
  Myers präsentieren, sowie
  Anwendungsmöglichkeiten, insbesondere im Vergleich mit
  Suffixtrees, beurteilen.

\noindent

\end{abstract}
%----------------------------------------------------------------------------------------
%	ARTICLE CONTENTS
%----------------------------------------------------------------------------------------

% \begin{multicols}{1} % Two-column layout throughout the main article text
\section{Einleitung}
Suffixbäume sind eine häufig genutzte Datenstruktur für
viele Anwendungsfälle des Pattern Matchings. Mit diesen eng
verwandt sind Suffixarrays. So können diese für einen
Großteil der Anwendungen jener Datenstruktur benutzt werden,
wobei der entscheidende Vorteil, der für die häufige Nutzung
der Suffixarrays verantwortlich ist, der geringere
Speicherbedarf ist (für beide gilt $\mathcal{O}(n)$, doch während
eine aktuelle Suffixbaum-Implementation 13-15 Bytes pro
Zeichen benötigt, kommen Suffixarrays aktueller
Implementierungen mit 5 Bytes aus)\footnote{SJ Puglisi, WF Smyth, AH Turpin: A Taxonomy of Suffix Array Construction
Algorithms}. \\
Die Grundidee des Pattern Matchings mit Suffixarrays ist es, einen vorhandenen
Text in alle möglichen Suffixe aufzuteilen. Diese Suffixe
werden, durch die Indizes ihrer Position im Text vertreten, in einem Array
lexikografisch sortiert gespeichert. Seit die Datenstruktur
1989 von Udi Manber und Gene Myers vorgeschlagen wurde, wurde eine große
Anzahl von Konstruktionsalgorithmen für Suffixarrays
entwickelt. Der Algorithmus von Manber und Myers wird hier
im Folgenden vorgestellt.
Auch sind in den letzten 25 Jahren viele Möglichkeiten des
Suchens eingeführt worden. Die naheliegende und für die
Einfachheit erstaunlich effiziente Möglichkeit
Binäre Suche zu nutzen wird in dieser Einführung zur
Darstellung genutzt.

\section{Konstruktion}
Analog zu Suffixtrees ist die größte Herausforderung für die Nutzung von
Suffixarrays in theoretischer Hinsicht die Entwicklung von Algorithmen zur
Konstruktion dieser Datenstruktur.
\subsection{Naiver Ansatz}
Eine erste mögliche Methode der Konstruktion eines Suffixarrays ist die Nutzung
eines bereits auf anderem Wege konstruierten Suffixtrees. Die Verwendung von
Tiefensuche auf dem Suffixtree liefert in Reihenfolge die erforderlichen Suffixe.

Dieser Ansatz widerspricht jedoch der Ratio für die Verwendung von Suffixarrays:
einem geringeren Bedarf an Speicher.
Wenn man also den Umweg über einen zuerst zu konstruierenden Suffixtree geht, so
geht eben dieser Vorteil, die Vermeidung des großen Speicherbedarfs, der für die
Konstruktion von Suffixtrees anfällt, verloren.
Es müssen also gesondert für die Konstruktion von Suffixarrays entwickelte
Algorithmen gefunden werden, um eine sinnvolle Nutzung dieser Datenstruktur in
der Praxis zu ermöglichen.
Im Folgenden wird der von Udi Manber und Gene Myers im Jahre 1989 veröffentlichte
Algorithmus vorgestellt werden

\subsection{Manber-Myers-Algorithmus}
Die Idee für den Konstruktionsalgorithmus ist die Folgende: Da die Suffixe aller
Suffixe des Suffixarrays auch wieder Suffixe des Suffixarrays sind, lässt sich
nach der Sortierung nach n Zeichen für die Sortierung nach 2n Zeichen ausnutzen,
dass nach den nächsten n Zeichen der zu sortierenden Suffixe bereits sortiert
wurde, nämlich die nächsten n Zeichen bereits sortierter Teil (Präfix) eines Suffix
sind.

Der Algorithmus nutzt für die Sortierung Bucket Sort, wobei in einem Bucket schrittweise
jene Suffixe aufbewahrt werden, die nach der Anzahl bereits sortierter Zeichen
lexikografisch in der gleichen Ordnung auftreten: Nach zwei sortierten Zeichen
wären z.B. die Suffixe "`te"' und "`tete"' im gleichen Bucket.

Für die Konstruktion bedarf es dreier Integerarrays Pos, Prm und Count, wobei Pos das
zu konstruierende Suffixarray und Prm die Inversion von Pos ist, und zweier
Booleanarrays, BH und B2H. Das Array BH hat an den Positionen eine 1 (True), die das Suffix, das
am weitesten links in einem Bucket ist, kennzeichnen. Die Arrays B2H und Count
sind temporäre Hilfsarrays, deren Funktion in der Folge erläutert wird.
Die Sortierung nach dem ersten
Zeichen und entsprechende Initialisierung der anderen Arrays kann mit Hilfe
von Radixsort in $\mathcal{O}(n)$ vorgenommen werden. \\


Nun soll der Schritt der Verdopplung der Anzahl der Zeichen, nach denen sortiert
wurde, erfolgen, wenn also bisher nach H Zeichen sortiert wurde, soll im nächsten
Schritt nach 2H Zeichen sortiert worden sein.
Zuerst wird für alle i Count[i] auf 0 gesetzt. Mit Hilfe des Arrays BH wird Prm[i]
für alle i auf das am weitesten linke Suffix des jeweiligen Bucket zeigen anstatt
der genauen Position des Suffix. Diese beiden Operationen können in $\mathcal{O}(n)$ absolviert werden.
Anschließend gehen wir der Reihe nach Pos ab, Bucket für Bucket. Seien l und r
die Abgrenzungen eines Buckets (den wir aus dem vorherigen Sortierschritt erhalten haben).
Sei $T_{i}$ Pos[i]-H, also ein Verweis auf jenes Suffix, das das mit Pos[i] referenzierte
Suffix weniger der ersten H Zeichen ist. Auf diese Art und Weise nutzen wir die
Tatsache aus, dass das Suffix mit der Anzahl H weniger Zeichen bereits nach den
ersten H Zeichen sortiert worden ist. Für jedes i, l $\le$ i $\le$ r, erhöhen wir Count[Prm[$T_{i}$]]
um 1, setzten Prm[$T_{i}$] = Prm[$T_{i}$] + Count[Prm[$T_{i}$]] - 1, und setzten B2H[Prm[$T_{i}$]]
auf 1.
Damit werden alle Suffixe deren Symbole von H + 1 bis 2H gleich dem Präfix der
Länge H des aktuellen Buckets sind an die Spitze ihres Buckets in Prm gesetzt.
Pos wird erst im Anschluss aktualisiert. Die Liste B2H markiert, wo Präfixe
bewegt worden sind. \\

Vor dem nächsten Bucket, das sortiert wird, gehen wir noch einmal über den Bucket,
finden alle bewegten Suffixe und setzen das Array B2H derart auf 0, so dass nur
noch die linkeste Position eines jeden neues Buckets der Sortierung nach 2H Zeichen
mit einer 1 markiert ist.

Zuletzt aktualisieren wir Pos, wobei wir nutzen, dass Prm die Inversion von Pos ist,
und setzen BH auf den Wert von B2H am Ende der Sortierung nach 2H Zeichen. \\

Diese Schritte können in $\mathcal{O}(n)$ durchgeführt werden. Es gibt höchstens $\log{} n$
Durchführungen der Verdopplung der sortierten Zeichen und darum kann die ganze
Sortierung des Suffixarrays in $\mathcal{O}(n \log{} n)$ ausgeführt werden mit dem Algorithmus
von Manber und Myers.

\begin{lstlisting} [frame=single]
if W <= S pos[0]:
    Lw = 0

else if W > S pos[N-1]:
    Lw = N

else:
    L, R = 0, N-1
    while R-L > 1:
        M = (L+R) / 2
        if W <= S pos[M]:
            R = M
        else:
            L = M
    Lw = R

for i in buckets[l,r]:
    Count[l] = 0
    for i in [l,r]:
        Prm[Pos[c]] = l

d = N - H
e = Prm[d]
Prm[d] = e + Count[e]
Count[e] = Count[e] + 1
B2H[Prm[d]] = True
for bucket[l,r]:#in Reihenfolge der bisher erfolgten Sortierung
    for d in (Pos[c]-H) where c in [l,r]:
        e = Prm[d]
        Prm[d] = e + Count[e]
        Count[e] = Count[e] + 1
        B2H[Prm[d]] = True

    for d in (Pos[c]-H) where c in [l,r]:
        if B2H[Prm[d]]:
            e = min(j) where j > Prm[d] and (BH[j] or not B2H[j])
            for f in [Prm[d] + 1, e - 1]:
                B2H[f] = False
for i in [0, N - 1]:
    Pos[Prm[i]] = i

for i in [0, N - 1]:
    BH[i] = B2H[i]


\end{lstlisting}

\section{Suche}
Ein für seine Einfachheit erstaunlich effizienter Suchalgorithmus soll hier
vorgestellt werden. Dieser stellt eine Anwendung des Prinzips der Binären Suche
auf Suffixarrays dar.
Sei Pos ein lexikografisch sortiertes Arrays der Suffixe eines Strings S, wobei
Pos[k] die Position des ersten Zeichen in S des k-kleinsten Suffix in der Menge
aller Suffixe angibt.
Wollen wir nun alle Vorkommen eines Strings W=w0 w1 w2 ... wp-1 der Länge P <= N
in S finden, so müssen wir das lexikografisch kleinste (Lw) und größte Suffix (Rw) finden,
für die W Präfix ist.
Lw findet man, indem man Binäre Suche anwendet auf Pos mit dem Suchwort W mit der
Suchrelation <= als Vergleichsoperation der jeweiligen Teilungsschritte des Arrays.
Für Rw analog mit >=.
Hat man Lw und Rw gefunden, so macht man sich zunutze, dass die Suffixe lexikografisch sortiert
sind und also alle zu findenden Suffixe in Pos zu finden sind, indem man ab Pos[Lw]
die Anzahl Rw-Lw+1 Suffixe als die gesuchten Einträge registriert.

Die Suche nach Lw und Rw hat jeweils O(log N) Vergleiche von Strins mit jeweils
O(P) Zeichenvergleichen, wobei P die Länge von W ist. Daher erlaubt uns das Suffixarray
Pos alle Vorkommen von W in einer Laufzeit von O(P*log N) zu finden.

\begin{lstlisting} [frame=single]
if W <= S pos[0]:
    Lw = 0

else if W > S pos[N-1]:
    Lw = N

else:
    L, R = 0, N-1
    while R-L > 1:
        M = (L+R) / 2
        if W <= S pos[M]:
            R = M
        else:
            L = M
    Lw = R


\end{lstlisting}




\section{Quellen}
Udi Manber, Gene Myers: "Suffix arrays: A new method for on-line string searches", 1989
Dan Gusfield: "Algorithms on Strings, Trees and Sequences: Computer Science and Computational Biology", 1997


% \end{multicols}

\end{document}
